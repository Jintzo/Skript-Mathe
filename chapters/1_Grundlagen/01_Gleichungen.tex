\chapter{Gleichungen}
\begin{inhalt}
  Lösen von Gleichungen mit
  \begin{itemize}
    \item Äquivalenzumformungen
    \item Mitternachtsformel
    \item Satz vom Nullprodukt
    \item Substitution
  \end{itemize}
\end{inhalt}

\ \\

Eine \emph{Gleichung} sagt aus, dass zwei Terme\sidenote{Ein \textbf{Term} ist ein Rechenausdruck, der aus Zahlen, Rechenzeichen und Klammern bestehen kann.} gleich sind. Meistens hängen diese
von einer Variable --- zum Beispiel $x$ --- ab. Es gilt herauszufinden, für welche Werte von $x$ die Gleichung erfüllt ist:
\begin{equation*}
  x^2+1=10\ \Rightarrow\ x=\pm 3, \text{ denn } {(\pm 3)}^2+1=9+1=10
\end{equation*}

\begin{bla}{Äquivalenzumformungen}
  \begin{marginfigure}[3em]
    \begin{tcolorbox}[colback=white!95!black,colframe=white!75!black,title=CAS:,arc=0mm]
      \begin{scriptsize}
        \textbf{Calculator}: \\*
        \menu{Menü > Algebra > Lösen}
        \begin{flushright}
          \( \text{\textsc{solve}}(5x+4 = 3x-2 \ , \ x) \) \\*
          \( \bm{\leadsto x = -3} \)
        \end{flushright}
      \end{scriptsize}
    \end{tcolorbox}
  \end{marginfigure}
  Bei einer \emph{Äquivalenzumformung} wird auf beiden Seiten der Gleichung die selbe Rechenoperation durchgeführt.
  \\
  \textbf{Achtung}: Wenn du auf beiden Seiten die Wurzel ziehst, dann vergiss nicht, dass jede Zahl zwei Wurzeln hat! \\
  \begin{equation*}
    \begin{alignedat}{4}
                        && 5x+4 & = 3x-2 & \quad & | -4 \\
      \Leftrightarrow\  && 5x   & = 3x-6 & \quad & | -3x \\
      \Leftrightarrow\  && 2x   & = -6   & \quad & | *\tfrac{1}{2} \\
      \Leftrightarrow\  && x    & = -3
    \end{alignedat}
  \end{equation*}
\end{bla}



\begin{bla}{Lösen von quadratischen Gleichungen: die Mitternachtsformel}
  \begin{marginfigure}[7em]
    \begin{tcolorbox}[colback=white!95!black,colframe=white!75!black,title=CAS:,arc=0mm]
      \begin{scriptsize}
        \textbf{Calculator}: \\*
        \menu{Menü > Algebra > Nullstellen}
        \begin{flushright}
          \( \text{\textsc{zeros}}(3x^2-3 \ , \ x) \) \\*
          \( \bm{\leadsto \{ -1, 1 \}} \)
        \end{flushright}
      \end{scriptsize}
    \end{tcolorbox}
  \end{marginfigure}
  Gleichungen der Form $ax^2+bx+c=0$ ($a,b$ und $c$ sind hierbei beliebige Zahlen, können also auch Null sein) können mit der \emph{Mitternachtsformel}
  gelöst werden. Man erhält \textbf{zwei} Ergebnisse $x_1$ und $x_2$, indem man für $\pm$ einmal $+$ und einmal $-$ rechnet:
  \begin{equation*}
    x_{1,2}=\frac{-b\pm \sqrt{b^2-4ac}}{2a}
  \end{equation*}
  Das funktioniert so:
  \begin{equation*}
    \begin{alignedat}{4}
                        && 2x          & = x^2-3                                    & \quad & |\ +3 \\
      \Leftrightarrow\  && 2x+3        & = x^2                                      & \quad & |\ -x^2 \\
      \Leftrightarrow\  && -x^2+2x+3   & = 0                                        & \quad & |\ \text{MNF:\ } a=-1, b=2, c=3 \\
      \Leftrightarrow\  && x_{1,2}     & = \frac{-2\pm \sqrt{2^2-4*(-1)*3}}{2*(-1)} & \quad & \\
      \Leftrightarrow\  && x_{1,2}     & = \frac{-2 \pm 4}{-2}                      & \quad & \\
    \end{alignedat}
  \end{equation*}
  Wir erhalten also $x_1 = -1$ und $x_2 = 3$.
\end{bla}



\begin{bla}{Satz vom Nullprodukt}
  Das Produkt von zwei Termen ist genau dann Null, wenn mindestens einer der beiden Terme Null ist.
  Liegt also eine Gleichung der Form
  \begin{equation*}
    (\text{Term}_1)*(\text{Term}_2)=0
    \end{equation*}
  vor, so können die beiden Terme einzeln betrachtet werden. Man betrachtet $(\text{Term}_1)=0$ und $(\text{Term}_2)=0$. Es gibt deswegen meistens zwei Möglichkeiten für $x$, die die Gleichung erfüllen:
  \begin{equation*}
    (x-1)(x-3) = 0
  \end{equation*}
  hat also die Lösungen $x=1$ und $x=3$, da genau dann einer der beiden Terme Null ist.
\end{bla}

\begin{bla}{Substitution}
  Manche Gleichungen lassen sich nicht durch die bisher bekannten Hilfsmittel lösen, zum Beispiel, weil Potenzen von $x$ mit einem Exponenten $>2$ vorkommen. Eine Möglichkeit solche Gleichungen zu lösen bietet das \emph{Substituieren}. Hierbei wird ein Term durch eine Variable abgekürzt und die
  so vereinfachte Gleichung gelöst. Danach macht man das Abkürzen rückwärts und erhält so die Ergebnisse der ursprünglichen Gleichung:
  \begin{align*}
    x^4+11x^2+10 & = 0 \\
    \text{Substitution:}\ z & = x^2 \\
    \rightsquigarrow z^2+11z+10 & = 0 \\
    \text{Man erhält mit der MNF}: z_1 & = -10, z_2=-1 \\
    \text{Rücksubstitution}\ x_1: x_1^2 & = z_1=-10 \\
    \text{Rücksubstitution}\ x_2: x_2^2 & = z_2=-1 \\
  \end{align*}
  Die Gleichung hat also keine Lösung, da es keine reellen Zahlen gibt, deren Quadrat kleiner Null ist.
\end{bla}
