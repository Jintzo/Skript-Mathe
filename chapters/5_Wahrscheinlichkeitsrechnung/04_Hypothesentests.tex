\chapter{Hypothesentests}
\begin{inhalt}
  \begin{itemize}
    \item Was sind Hypothesentests?
    \item Einseitige Hypothesentests
  \end{itemize}
\end{inhalt}

\begin{bla}
  {Hypothesentests}
  Für ein gegebenes Bernoulli-Experiment stellt sich in der echten Welt oft die Frage, wie hoch die Wahrscheinlichkeit $p$ für einen Treffer wirklich ist.
  Nachprüfen kann man dies allerdings nur in stichprobenartigen Tests.

  Da so ein Experiment aber zufällig verläuft, kann das Ergebnis von dem Erwartungswert abweichen (zum Beispiel kann beim Münzwurf tausendmal nacheinander Kopf fallen, auch wenn die Münze mit $p=0.5$ perfekt funktioniert), also kann man aus einer kleinen Stichprobe nicht garantiert auf die Wahrscheinlichkeit der echten Welt schließen.
  Im Fall der Münze könnte man aber vermuten, dass sie kaputt ist.
  Das Ziel ist es deswegen, eine Aussage zu machen, von der man sich sehr sicher ist: ``Ich bin mir sehr sicher ($95\%$), dass ich mit mehr als \( 50\% \) Kopf werfe.''

  Mathematisch ist also gesucht:
  Ein alternative Vermutung für die Trefferwahrscheinlichkeit (hier $p>0.5$) und zusätzlich die Wahrscheinlichkeit, dass man falsch liegt und die echte Welt doch anders aussieht (hier $\sigma = 0.05$). Das nennt man das Signifikanzniveau.
\end{bla}

\begin{bla}{Nullhypothese und Alternative}
  Mathematisch baut man das so auf:
  Man beginnt mit der sogenannten Nullhypothese $H_0$ und nimmt zunächst einmal für das Experiment feste Werte an.
  Dabei wird auch eine Wahrscheinlichkeit für Treffer angenommen, z.B. $p=0.5$, damit man überhaupt etwas ausrechnen kann.

  Dann formuliert man die \emph{Alternative} $H_1$, zum Beispiel die Vermutung $p>0.5$.
\end{bla}

\begin{bla}
{Testen: Testumfang, Annahme- und Ablehnungsbereich}
  Man muss sich jetzt entscheiden, ob die Welt so funktioniert wie die Nullhypothese es aussagt, oder eben nicht.

  Dazu führt man einen Test mit einem gewissen \emph{Testumfang} $n$ durch.

  Dann rechnet man: Wenn das System funktionieren würde, wie in $H_0$ behauptet, kann man einen Bereich an Ergebnissen finden, der sehr unwahrscheinlich getroffen wird (Zum Beispiel mehr als \( 530 \) Mal `Kopf' von \( 1000 \) Münzwürfen ist seltener als \( 3 \% \)).

  Wenn das Experiment jetzt also mehr als \( 530 \) Köpfe liefert, kann man mit hoher Wahrscheinlichkeit (\( 97 \% \)) sagen, dass die Münze nicht richtig funktioniert.
  Man nennt den Bereich, in dem Ergebnisse liegen müssen, damit man die Nullhypothese verwirft, den \emph{Ablehnungsbereich}, der Rest ist der \emph{Annahmebereich}.
\end{bla}

\section{Einseitiger Hypothesentest}

\begin{bla}{Beispiel --- rechtsseitiger Hypothesentest}
  \begin{marginfigure}
    \begin{tcolorbox}[colback=white!95!black,colframe=white!75!black,title=CAS:,arc=0mm]
      \begin{scriptsize}
        \textbf{1. Calculator}: \\*
        \menu{Menü > Wskt > Verteilung > binomCdf} \\*
        \( \leadsto f(x) := \) \\* \hfill \( \text{\textsc{binomCdf}}(300, \tfrac{1}{6}, 0, \text{\textsc{int}}(x)) \) \\*
        (\textsc{int} rundet \( x \) auf die nächste ganze Zahl herab, weil wir ja nur ganze Male würfeln können) \\*
        \ \\*
        \textbf{2. Graph}: \\*
        \( f1(x) = f(x) \) liefert den gesuchten Graph, \\*
        \menu{Menü > Tabelle > Tabelle mit \dots} \ die gesuchte Tabelle
      \end{scriptsize}
    \end{tcolorbox}
  \end{marginfigure}
  Wir haben einen normalen Würfel, vermuten aber, dass er mehr Sechsen würfelt als er sollte.
  \begin{enumerate}
    \item \textbf{Nullhypothese}: $H_0: p=\tfrac{1}{6}$, \textbf{Alternative}: $H_1: p>\tfrac{1}{6}$
    \item \textbf{Signifikanzniveau}: Wir legen ein Signifikanzniveau von $5\%$ fest.
    \item \textbf{Stichprobenumfang}: Wir werfen den Würfel $n=300$ mal.
    \item \textbf{Annahmebereich}: Der Annahmebereich ist $A=[0,b]$. $b$ ist die kleinste Zahl, für die noch $P(X \leq b)>95\%$ ist ($X$ ist die Zahl der Sechsen bei $n$ Würfen). Wir lassen hierfür vom CAS den Graphen von
    \begin{equation*}
      \text{\textsc{binomPdf}}\left( 300, \tfrac{1}{6}, 0, \text{\textsc{int}}(x) \right)
    \end{equation*}
    zeichnen und anschließend die Wertetabelle angeben. Nun können wir $b$ einfach ablesen.
    \item \textbf{Irrtumswahrscheinlichkeit}: Die Irrtumswahrscheinlichkeit ist die Wahrscheinlichkeit, dass der Ausgang im abgelehnten Bereich liegt. Dieser ist hier $(b,300]$, also ist die Irrtumswahrscheinlichkeit $P(X > b)=P(X \geq (b+1))$.
  \end{enumerate}
  Wir erhalten $b=61$ und eine Irrtumswahrscheinlichkeit von $P(X \geq 62)=0.0402=4.02\%$. Würfelt man also bei einem Test $65$ Sechsen, so wird die Hypothese abgelehnt.
\end{bla}

\begin{bla}{Beispiel --- linksseitiger Hypothesentest}
  \begin{marginfigure}
    \begin{tcolorbox}[colback=white!95!black,colframe=white!75!black,title=CAS:,arc=0mm]
      \begin{scriptsize}
        \textbf{1. Calculator}: \\*
        \menu{Menü > Wskt > Verteilung > binomCdf} \\*
        \( \leadsto f(x) := \) \\* \hfill \( \text{\textsc{binomCdf}}(300, \tfrac{1}{6}, \text{\textsc{int}}(x), 300) \) \\*
        \ \\*
        \textbf{2. Graph}: \\*
        \( f1(x) = f(x) \) liefert den gesuchten Graph, \\*
        \menu{Menü > Tabelle > Tabelle mit \dots} \ die gesuchte Tabelle
      \end{scriptsize}
    \end{tcolorbox}
  \end{marginfigure}
  Wir haben einen normalen Würfel, vermuten aber, dass er weniger Sechsen würfelt als er sollte.
  \begin{enumerate}
    \item \textbf{Nullhypothese}: $H_0: p=\tfrac{1}{6}$, \textbf{Alternative}: $H_1: p<\tfrac{1}{6}$
    \item \textbf{Signifikanzniveau}: Wir legen ein Signifikanzniveau von $5\%$ fest.
    \item \textbf{Stichprobenumfang}: Wir werfen den Würfel $n=300$ mal.
    \item \textbf{Annahmebereich}: Der Annahmebereich ist $A=[b,300]$. Hier ist $b$ die kleinste Zahl, für die noch $P(X \leq b) > 5\%$ ist. Wir ermitteln diese wieder mit dem Taschenrechner.
  \end{enumerate}
  Wir erhalten mit dem Taschenrechner $b=40$ und eine Irrtumswahrscheinlichkeit von $P(X \leq 39)=0.0486=4.86\%$. Würfelt man also bei einem Test $35$ Sechsen, so wird die Hypothese abgelehnt.
\end{bla}

Ende.
