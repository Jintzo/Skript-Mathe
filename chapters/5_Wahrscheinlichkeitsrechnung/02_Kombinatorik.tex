\chapter{Kombinatorik}
\begin{inhalt}
  \begin{itemize}
    \item Multiplikationsformel
    \item Urnenmodelle
  \end{itemize}
\end{inhalt}
\section{Multiplikationsformel}

\begin{bla}{Fakultät}
  \begin{marginfigure}
    \begin{tcolorbox}[colback=white!95!black,colframe=white!75!black,title=CAS:,arc=0mm]
      \begin{scriptsize}
        \textbf{Calculator}: \\*
        \menu{Menü > Wahrscheinlichkeit > Fakultät}
      \end{scriptsize}
    \end{tcolorbox}
  \end{marginfigure}
  Für $n\in\mathbb{N}$ ist $n!=n*(n-1)*\dots *1$.
\end{bla}

\begin{bla}{Multiplikationsformel}
  Die Multiplikationsformel stellt die Basis der Kombinatorik dar. Sie sagt aus:
  \\
  Möchte man $n$ Dinge auf $k$ Stellen verteilen, so gibt es für die erste Stelle
  $n$ Möglichkeiten, für die zweite $n-1$ Möglichkeiten,\dots und für die letzte $n-k$
  Möglichkeiten.
  \\
  \textbf{Beispiel}: Möchte man seine $7$ Sockenpaare auf die $7$ Plätze für Socken
  im Schrank verteilen, so gibt es dafür $7*6*5*4*3*2*1=7!=5040$ Möglichkeiten.
  Insbesondere gilt also, dass es $n! $ Möglichkeiten gibt, $n$ Dinge anzuordnen.
\end{bla}

\clearpage

\section{Urnenmodell}

Um kombinatorische Zusammenhänge veranschaulichen zu können werden häufig
\emph{Urnenmodelle} verwendet.

\begin{bla}{Ziehen mit vs.\ ohne Zurücklegen}
  Wir stellen uns eine Urne vor, also ein undurchsichtiges Gefäß, in dem
  verschiedenfarbige Kugeln sind.
  Wir haben zwei Möglichkeiten, wie wir ihr Kugeln entnehmen können:
  \begin{itemize}
    \item \textbf{Ziehen mit Zurücklegen}: Wir ziehen eine Kugel, merken uns ihre
    Farbe und legen sie wieder zurück. Diesen Vorgang können wir dann beliebig
    oft wiederholen. Hier beeinflussen sich die einzelnen Ziehungen offensichtlich nicht.

    \item \textbf{Ziehen ohne Zurücklegen}: Wir ziehen eine Kugel, merken uns ihre
    Farbe und legen sie beiseite. Diesen Vorgang kann man nur so oft wiederholen wie
    Kugeln in der Urne sind. Man kann übrigens alternativ auch mehrere Kugeln auf einmal
    ziehen (das ist identisch zum Ziehen ohne Zurücklegen). Offensichtlich beeinflussen sich die einzelnen Ziehungen gegenseitig.
  \end{itemize}
\end{bla}
