\chapter{Winkel und Spiegelungen}
\begin{inhalt}
  Winkel zwischen
  \begin{itemize}
  	\item Vektoren,
    \item Geraden,
    \item Ebenen,
    \item Gerade und Ebene
  \end{itemize}
%  Spiegelung von
%  \begin{itemize}
%  	\item Punkt, Gerade und Ebene an Gerade,
%  	\item Punkt, Gerade und Ebene an Ebene.
%  \end{itemize}
\end{inhalt}

Für die Winkel zwischen Objekten schreiben wir $\angle(\text{Ding}_1,\text{Ding}_2)$.

\begin{bla}{Skalarprodukt}
	Aus dem Skalarprodukt zwischen zwei Vektoren ergibt sich eine Winkelbeziehung:
	\[
	\vec p * \vec q = |\vec p| * |\vec q| * \cos\left(\angle\left(\vec p, \vec q\right)\right)
	\]

	Einfacher schreibt man oft:
	\[
	\frac{\vec p * \vec q }{ |\vec p| * |\vec q| } = \cos\left(\angle\left(\vec p, \vec q\right)\right)
	\]
\end{bla}

\begin{bla}{Kreuzprodukt}
	Das Kreuzprodukt funktioniert ähnlich:
	\[
	| \vec p \times \vec q | = |\vec p| * |\vec q| * \sin\left(\angle\left(\vec p, \vec q\right)\right)
	\]
	Da das Kreuzprodukt einen Vektor liefert, muss man hier noch den Betrag ausrechnen.
\end{bla}

\clearpage

\begin{bla}{Geraden}
	Der Winkel zwischen zwei Geraden ist der Winkel zwischen ihren Richtungsvektoren.

	Wenn sich dabei ein Winkel größer als \SI{90}{\degree} ergibt, muss man für ein sinnvolles Ergebnis diese \SI{90}{\degree} wieder abziehen.
	Alternativ kann man auch $$| \vec p * \vec q |$$ \emph{mit Betragsstrichen} bei der Rechnung verwenden, dann ergibt sich automatisch der kleinere Winkel.
\end{bla}

\begin{bla}{Ebenen}
	Der Winkel zwischen zwei Ebenen ist genau der Winkel zwischen ihren Normalenvektoren:
	\[
	\angle(E_1,E_2) = \angle(n_1,n_2)
	\]
\end{bla}

\begin{bla}{Gerade-Ebene}
	Man bekommt den Winkel zwischen einer Geraden und einer Ebenen, indem man den zwischen dem Richtungsvektor der Geraden und dem Normalenvektor der Ebenen berechnet.
	Diesen Winkel muss man noch von \SI{90}{\degree} abziehen.

	Man kann sich den zweiten Rechenschritt sparen, indem man statt eines Cosinus einen Sinus verwendet:
	\begin{align*}
	\angle(g,E) = \SI{90}{\degree} - \angle(\vec u_g, \vec n_E)
	&= \SI{90}{\degree} - \cos^{-1} \left( \frac {\vec u_g * \vec n_E }{ |\vec u_g| * |\vec n_E| } \right)
	\\
	&= \sin^{-1} \left( \frac {\vec u_g * \vec n_E }{ |\vec u_g| * |\vec n_E| } \right)
	\end{align*}
\end{bla}

\begin{bla}
	{Spiegelungen}
	Für eine Spiegelung eines Punktes an etwas anderem muss zuerst der Spiegelpunkt gefunden werden.
	\\
	Spiegelt man also einen Punkt $P$ an einer Geraden oder Ebene, möchte man darauf einen Punkt $F$ finden, so dass $\overrightarrow{PF}$ senkrecht auf der Ebene oder Geraden steht.

	Zum Spiegelpunkt gelangt man jetzt, indem man von $P$ zweimal die Strecke $\overrightarrow{PF}$ läuft:
	\[
	\vec{p'} = \vec{p} + 2* \overrightarrow{PF} = \vec{f} + \overrightarrow{PF}
	\]
\end{bla}
