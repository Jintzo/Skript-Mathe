\chapter{Wachstum}
\begin{inhalt}
  \begin{itemize}
    \item exponentielles und beschränktes Wachstum
  \end{itemize}
\end{inhalt}

\section{Exponentielles Wachstum}

Etwas wächst \emph{exponentiell}, wenn es von einem zum nächsten Schritt um einen bestimmten Faktor wächst:
\begin{equation*}
  f(x+1)=b*f(x).
\end{equation*}
Wir modellieren einen exponentiellen Wachstumsprozess mit einer \emph{Exponentialfunktion}.

\begin{bla}{Startwert}
  Der \emph{Startwert} gibt an, welchen Wert der Wachstumsprozess am Anfang, also in $f(0)$ hat. Es ist $e^0=1$, also ist $a*e^0=k*1=a$. Also hat die Funktion $f(x)=a*e^x$ den Startwert $a$.
\end{bla}

\begin{bla}{Wachstumskonstante}
  Die \emph{Wachstumskonstante} gibt an, wie schnell das exponentielle Wachstum ist. Wie zur Änderung der Periode bei trigonometrischen Funktionen bekommt auch hier $x$ einen Faktor, der für eine Streckung/Stauchung des Graphen sorgt: $f(x)=a*e^{kx}$ hat den Startwert $a$ und die Wachstumskonstante $k$.
\end{bla}

\begin{bla}{Bedeutung der Wachstumskonstante}
  Wir haben gesagt, dass bei einem exponentiellen Wachstumsprozess $f(x+1)=b*f(x)$ ist, d.h.
  \begin{align*}
    f(x+1) &= f(x)*b \\
    f(x+2) &= f(x+1)*b = f(x)*b^2 \\
    f(x+3) &= f(x+2)*b = f(x)*b^3 \\
    \cdots
  \end{align*}
  Also ist $f(x)=f(0)*b^x=f(0)*e^{\ln(b)x}$ mit Startwert $f(0)$.
\end{bla}

\begin{bla}{Bestimmen des Wachstumsfaktors}
  Ist eine Datenmenge gegeben, so kann man jeweils den Wachstumsfaktor von einem Schritt zum anderen bestimmen:
  \begin{equation*}
    f(x+1)=b*f(x) \Leftrightarrow b=\frac{f(x+1)}{f(x)}.
  \end{equation*}
\end{bla}

\begin{bla}{Bestimmen des Startwerts}
  Ist eine Datenmenge gegeben, so ist der Startwert der Datenwert an der Nullstelle.
\end{bla}

\begin{bla}{Verdoppelungs-/Halbwertszeit}
  \begin{itemize}
    \item \textbf{Verdoppelungszeit.} Die Verdoppelungszeit ist die Zeit $x_0$, für die $f(x_0)=2*f(0)$ gilt. Man erhält sie durch Auflösen dieser Gleichung nach $x_0$.
    \item \textbf{Halbwertszeit.} Die Halbwertszeit ist die Zeit $x_0$, für die $f(x_0)= \tfrac{1}{2}*f(0)$ gilt. Man erhält sie durch Auflösen dieser Gleichung nach $x_0$.
  \end{itemize}
\end{bla}

\section{Beschränktes Wachstum}

Wächst etwas, aber mit zunehmender Zeit immer langsamer, so liegt \emph{beschränktes Wachstum} vor. Wichtig ist hier vor allem die

\begin{bla}{Schranke}
  Ein beschränkt wachsender Bestand ist nach oben durch eine \emph{Schranke} $S$ beschränkt.
\end{bla}

\begin{bla}{Beschränktes Wachstum}
  Ein beschränktes Wachstum kann beschrieben werden durch $f(x)=S-ce^{-kx}$. $S$ ist die Schranke, $c=S-f(0)$ ist der angepasste Startwert und $k=-\ln(b)$ der angepasste Wachstumsfaktor. Beim beschränkten Wachstum ist $0<a<1$.
\end{bla}

\begin{marginfigure}
  \begin{tikzpicture}[
      scale=0.7,
      thick,
      >=stealth',
      dot/.style = {
        draw,
        fill = white,
        circle,
        inner sep = 0pt,
        minimum size = 4pt
      }
    ]
    \coordinate (O) at (0,0);

    \draw[step=1cm,gray!40] (-0.1,-0.1) grid (3.9,3.9);

    % Achsen
    \draw[->] (-0.2,0) -- (4,0) coordinate[label = {below:$x$}] (xmax);
    \draw[->] (0,-0.1) -- (0,4) coordinate[label = {right:$f(x)$}] (ymax);

    % Graph
    \draw[domain=-0.2:4.2,smooth,variable=\x,red, label={right:$f$}] plot ({\x},{3-2.5*exp(-0.7*\x)});

    % Schranke
    \draw[domain=-0.1:4.5,dashed,black!60!green] plot ({\x},{3});
  \end{tikzpicture}
  \caption{Der Graph von $f(x)=3-2.5e^{-0.7x}$.}
\end{marginfigure}

\begin{bla}{Modellierung einer Wachstumsfunktion bei gegebenen Daten}
  $f(x)=S-ce^{-kx}$ mit
  \begin{enumerate}
    \item \textbf{Schranke}: Diese muss man häufig schätzen. Sind die Werte am Ende der Tabelle zum Beispiel $298.5 - 299.3 - 299.6 - 299.7 - \dots$, so ist $300$ wahrscheinlich die gesuchte obere Schranke.
    \item \textbf{c}: $c$ erhält man, in dem man den Startwert von $S$ abzieht: $c=S-f(0)$.
    \item \textbf{k}: $k$ erhält man, in dem man (wie beim exponentiellen Wachstum) $b$ berechnet. Dafür berechnet man $\frac{f(x+1)}{f(x)}$ für jedes angegebene $x$ und erhält so die durchschnittliche Wachstumskonstante $b$. Dann ist $k=-\ln(b)$.
  \end{enumerate}
\end{bla}

\begin{koch}
  \textbf{Modellierung eines exponentiellen Wachstums.} \\
  $f(x)=f(0)*e^{\ln(b)x}$ mit
  \begin{itemize}
    \item Startwert $f(0)$,
    \item Wachstum pro Schritt $b$.
  \end{itemize}
  \textbf{Modellierung eines beschränkten Wachstums.} \\
  \( f(x) = S - ce^{-kx} \) mit
  \begin{itemize}
    \item Schranke \( S \)
    \item ``Startwert'' \( c = S - f(0) \)
    \item angepasster Wachstumsfaktor \( k = -\ln(b) \) (\( b \) berechnet sich wie beim exponentiellen Wachstum)
  \end{itemize}
\end{koch}
