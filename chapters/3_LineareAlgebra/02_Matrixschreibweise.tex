\chapter{Matrixschreibweise}
\begin{inhalt}
  \begin{itemize}
    \item LGS als Matrix darstellen
    \item LGS-Umformungen bei der Matrixschreibweise
  \end{itemize}
\end{inhalt}

\leqnomode

\begin{bla}{Matrixschreibweise}
Eine Methode, die viele Schreibarbeit zu umgehen, ist die sogenannte Matrixschreibweise.
\begin{alignat*}{5}
  \LIN{I}{1}{2}{3}{4}{}
  \LIN{II}{5}{6}{7}{8}{}
  \LIN[0pt]{III}{1}{2}{4}{5}{}
\end{alignat*}
Dieses LGS kann man kurz schreiben als:
\begin{align*}
  \begin{pmatrix}
    1 & 2 & 3 \\
    5 & 6 & 7 \\
    1 & 2 & 4 \\
  \end{pmatrix}
  *
  \begin{pmatrix}
    x\\y\\z\\
  \end{pmatrix}
  =
  \begin{pmatrix}
    4\\8\\5
  \end{pmatrix}
\end{align*}
Man lässt also die Koeffizienten und das Ergebnis an den gleichen Stellen stehen und schreibt dazwischen die Variablen als Vektor.

Wenn man diese Matrix wieder als LGS schreiben möchte, muss man bedenken, dass die Variablen nicht zu den Zeilen gehören, sondern dass man den Vektor wieder kippen muss.
\end{bla}


\begin{bla}{Erweiterte Matrixschreibweise}
Noch schöner zum Rechnen ist die erweiterte Matrixdarstellung, bei der die Variablen einfach weggelassen werden:
\begin{align*}
  \left(\left.\begin{matrix}
    1 & 2 & 3 \\
    5 & 6 & 7 \\
    1 & 2 & 4 \\
  \end{matrix}\right|
  \begin{matrix}
    4\\8\\5\\
  \end{matrix}\right)
\end{align*}
Die erlaubten Umformungen kann man übersichtlich in den einzelnen Zeilen durchführen.
\end{bla}


\section{Lösungen}

\begin{bla}{Lösungsmenge}
  Löst man ein LGS, dann gibt man das Ergebnis als \emph{Menge} an.
  Man schreibt dazu alle Werte als Vektor in geschweifte Klammern und nennt das ganze \emph{Lösungsmenge L}:
  \begin{eqnarray*}
    L = \left\{
    \begin{pmatrix}
      x \\y \\ z
    \end{pmatrix}
    =
    \begin{pmatrix}
      2 \\ 1 \\ \frac{1}{2}
    \end{pmatrix}
    \right\}
    =
    \left\{\begin{pmatrix}
    2 \\ 1 \\ \frac{1}{2}
    \end{pmatrix}\right\}
  \end{eqnarray*}

\end{bla}

\begin{bla}{Nicht-eindeutige Lösungen}
  Die Lösung eines LGS muss nicht immer eindeutig sein, auch unendlich viele oder gar keine Lösungen können vorkommen.

  Betrachtet man zum Beispiel die Gleichungen
  \begin{align*}
    x + y &= 1 \tag{I}\\
    x + y &= 0 \tag{II}\\
  \end{align*}
  dann gibt es keine Lösungen.
  Man sagt die Lösungsmenge ist \emph{leer} und schreibt $L = \{\}$.

  Ein Beispiel für ein LGS mit unendlich vielen Lösungen ist
  \begin{align*}
    x - y = 0 \tag{I} \\
    2x - 2y = 0 \tag{II} \\
  \end{align*}
  Die Lösung dieser Gleichungen ist $x=y$, also gibt es unendlich viele $x$ und $y$, die das System lösen.
  Man nennt die Lösungsmenge \emph{unendlich} und schreibt sie als
  \begin{eqnarray*}
    L =
    \left\{ \begin{pmatrix}
      1 \\ 1
    \end{pmatrix} * k \text{, mit $k\in\R$} \right\}
  \end{eqnarray*}
  Im Teil "`Analytische Geometrie"' werden wir zeigen, was dies bedeutet und wie man diese Menge aufschreibt.

  Wenn man eine eindeutige Lösung sucht, dann braucht man mindestens soviele Gleichungen wie man Variablen hat.

\end{bla}

\begin{bla}{Stufenform: Keine Lösung}
  Ist ein LGS nicht lösbar, kann man das einfach an der Stufenform erkennen.

  Hat ein LGS keine Lösung, dann liegt das daran, dass zwei Gleichungen gegensätzliche Dinge aussagen:
  \begin{alignat*}{5}
    \LIN{Ia}{0}{1}{25}{10}{}
    \LIN[0pt]{IIa}{0}{1}{25}{99}{}
  \end{alignat*}
  als Matrix geschrieben ergibt sich:
  \begin{align*}
    \left(\left.\begin{matrix}
      1 & 25 \\
      1 & 25 \\
    \end{matrix}\right|
    \begin{matrix}
      10\\99\\
    \end{matrix}\right)
    &\text{ also folgt }
    \left(\left.\begin{matrix}
      1 & 25 \\
      0 & 0 \\
    \end{matrix}\right|
    \begin{matrix}
      10\\89\\
    \end{matrix}\right)\\
    \intertext{Die zweite Zeile entspricht der Gleichung}
    0*y + 0*z = 89
  \end{align*}
  Die Gleichung ist offensichtlich falsch, also gibt es keine Lösung.
\end{bla}

\begin{bla}{Stufenform: Unendliche Lösungen}
  Falls beim Lösen eines LGS weniger Gleichungen als Variablen übrig bleiben, kann man das LGS nicht mehr eindeutig lösen. So zum Beispiel:
  \begin{align*}
    \left(\left.
    \begin{matrix}
      1 & 0 & 2 \\
      0 & 1 & 1 \\
      1 & 1 & 3 \\
    \end{matrix}
    \right|
    \begin{matrix}
      5\\4\\9\\
    \end{matrix}
    \right)
    &\text{- umgeformt:}
    \left(\left.
    \begin{matrix}
      1 & 0 & 2 \\
      0 & 1 & 1 \\
      0 & 0 & 0 \\
    \end{matrix}
    \right|
    \begin{matrix}
      5\\4\\0\\
    \end{matrix}
    \right)\\
    \intertext{Die letzte Gleichung, $0=0$ stimmt zwar immer, bringt uns aber nichts. Uns bleiben also nur zwei Gleichungen für drei Variablen:}
    x    + 2z &= 5\\
       y +  z &= 4\\
    \intertext{Dieses Problem ist zwar nicht eindeutig lösbar, aber sobald man $z$ kennt, kann man $x$ und $y$ problemlos bestimmen - das kann man aber ausnutzen:}
    x    &= 5 - 2z\\
       y &= 4 -  z\\
    \intertext{Für irgendwelche $z$ kann man jetzt also $x$ und $y$ finden.
    Um die Lösung schön aufzuschreiben, führt man jetzt ein $t = z$ ein:}
    x &= 5 - 2t\\
    y &= 4 -  t\\
    z &= t\\
    \intertext{Jetzt kann man die Lösungsmenge einfach mit unbekanntem $t$ aufschreiben}
    L &= \left\{
    \begin{pmatrix}
      x\\y\\z
    \end{pmatrix}
    =
    \begin{pmatrix}
      5-2t\\4-t\\t
    \end{pmatrix}
    =
    \begin{pmatrix}
      5\\4\\0
    \end{pmatrix}
    +t*
    \begin{pmatrix}
      -2\\-1\\1
    \end{pmatrix}
    \right\}
  \end{align*}
\end{bla}
\reqnomode
