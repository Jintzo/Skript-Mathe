\chapter{Lineare Gleichungssysteme}
\begin{inhalt}
  \begin{itemize}
    \item Was sind lineare Gleichungssysteme?
    \item Umformen und Lösen von LGS
    \item Fehlerquellen
  \end{itemize}
\end{inhalt}
\leqnomode

\begin{bla}{Lineare Gleichungssysteme}
   \begin{marginfigure}[5em]
    \begin{tcolorbox}[colback=white!95!black,colframe=white!75!black,title=CAS:,arc=0mm]
      \begin{scriptsize}
        \textbf{Calculator}: \\*
          \menu{Menü > Algebra > GLS lösen > GLS lösen > Anz. d. Gleichungen und Variablen eingeben}
      \end{scriptsize}
    \end{tcolorbox}
  \end{marginfigure}
  In diesem Kapitel betrachten wir Gleichungen der Art
  \begin{alignat*}{5}
    \LIN[0pt]{A}{2}{-3}{2}{0}{}
  \end{alignat*}
  Solche Gleichungen nennt man \emph{linear}, wenn Variablen nur als $x^1$ oder $x$, aber nicht als $x^2, \sqrt{x}$ oder $x^y$ vorkommen.

  Aus ihr erhält man für die drei Variablen keine eindeutige Lösung wie $x=3$, $y=2$ und $z=275$ - dafür müsste pro Variable eine
  Gleichung vorhanden sein.
  Fügen wir nun zwei weitere Gleichungen hinzu, so erhält man beispielsweise:
  \begin{alignat*}{5}
    \LIN{I}{2}{-3}{2}{0}{}
    \LIN{II}{5}{6}{-2}{5}{}
    \LIN[0pt]{III}{-12}{3}{-1}{0}{}
  \end{alignat*}
  Die verwendeten Variablen sollen natürlich in allen drei Gleichungen denselben Wert haben.
  Man spricht hierbei von einem \emph{linearen Gleichungssystem} oder \emph{LGS}.
\end{bla}

\clearpage

\section{Rechnen mit LGS}

Natürlich möchte man gerne wissen, welchen Wert die Variablen haben, dazu muss man es also \emph{lösen}:

\begin{bla}{Stufenform}
  Sehr einfach zu lösen sind LGS, die \emph{Stufenform} haben, also zum Beispiel:
  \begin{alignat*}{5}
    \LIN{I}{2}{-3}{7}{3}{}
    \LIN{II}{0}{5}{-2}{5}{}
    \LIN[0pt]{III}{0}{0}{-1}{0}{}
  \end{alignat*}
  In Stufenform kann man eine Variable sofort ablesen, die anderen ergeben sich dann durch Einsetzen.

  Im LGS oben würde man zum Beispiel sehen, dass $z=0$ sein muss.
  In Gleichung II setzt man das ein und formt zu $y = 1$ um.
  Diese zwei bekannten Zahlen setzt man dann in die erste Gleichung ein und es ergibt sich $x=3$.
\end{bla}

\begin{bla}{Erlaubte Rechnungen}
  Um ein LGS in Stufenform zu bringen, hat man folgende Möglichkeiten:
  \begin{itemize}
    \item {\bfseries Multiplikation mit Zahlen}
    \\
      In einer Gleichung werden alle Terme mit einer Zahl multipliziert.
      Diese Zahl darf nicht die Null sein! (siehe "`\ref{LA:LGS:Fehlerquellen} Fehlerquellen"')
    \item {\bfseries Addition/Subraktion von Gleichungen}
    \\
      Zu einer Gleichung wird eine der anderen Gleichungen addiert oder subtrahiert.
    \item {\bfseries Vertauschen von Gleichungen}
    \\
      Aus Schönheitsgründen kann man die Reihenfolge der Gleichungen ändern.
  \end{itemize}
  Diese Umformungen sind recht einfach, daher gibt es auch nur wenige gefährliche Fehler, die man beim Lösen machen kann.
\end{bla}

\begin{bla}{Fehlerquellen}
\label{LA:LGS:Fehlerquellen}
Es ist wichtig, beim Umformen keine Informationen zu verhäckseln!\\
Schlechte Ideen sind:

\begin{itemize}
  \item \textbf{FEHLER: Multiplikation mit Null:} %Beispiel entfernt, da schlecht verständlich..
    Eine Gleichung eines linearen Gleichungssystems darf niemals mit $0$ multipliziert werden, da dabei sämtliche Informationen der Gleichung verloren gehen würden. Man würde effektiv ein Gleichungssystem mit drei Gleichungen in eins mit zwei verwandeln, doch damit kann man nicht mehr nach allen Variablen auflösen.
  \item \textbf{FEHLER: Verlieren einer Gleichung:}
    Sei zum Beispiel dieses einfache Gleichungssystem gegeben:
    \begin{alignat*}{5}
      \LIN{Ia}{1}{0}{0}{1}{}
      \LIN{IIa}{0}{1}{0}{1}{}
      \LIN[10pt]{IIIa}{0}{0}{1}{1}{Dieses LGS ist gelöst!}
    \hline
      \LIN{Ia}{1}{0}{0}{1}{}
      \LIN{IIa}{0}{1}{0}{1}{}
      \LIN[0pt]{IIIb}{1}{1}{0}{2}{IIIb = Ia + IIa}
    \end{alignat*}
    Hierbei sind sämtliche Informationen über $z$ verloren gegangen, die eindeutige Lösung vom Anfang damit auch, weil $z$ nicht mehr bestimmt werden kann!
  \item \textbf{FEHLER: Ungeschicktes Addieren von Gleichungen:}
    \begin{alignat*}{5}
      \LIN{Ia}{0}{1}{0}{1}{}
      \LIN[10pt]{IIa}{0}{0}{1}{1}{Die Lösung ist noch eindeutig}
      \hline
      \LIN{Ib}{0}{1}{1}{2}{Ib = Ia + IIa}
      \LIN[0pt]{IIb}{0}{1}{1}{2}{IIb = Ia + IIa}
    \end{alignat*}
    Dieses LGS hat plötzlich keine eindeutige Lösung mehr.
    Um diesen Fehler zu vermeiden, muss man etwas aufpassen.

    Wenn man sauber rechnet, macht man immer nur einen Umformungsschritt und schreibt dann das ganze neue LGS einmal auf.
    Mit den neuen Gleichungen wird dann weitergerechnet.

    Beim Rechnen hat natürlich niemand Zeit, so viel zu schreiben, deswegen muss man auf diesen Fehler achten.

    Im Beispiel oben wurde zuerst folgendes gemacht: Ib = Ia + IIa.
    Wenn man jetzt ganz ausführlich weitermacht, schreibt man das neue LGS auf, also die Gleichungen Ib und IIa.
    Versucht man jetzt die zweite Umformung, IIb = Ia + IIa, bemerkt man, dass
    die Gleichung Ia gar nicht mehr im LGS steht.
    Man darf also nur noch mit Ib weiterrechnen.

    \textbf{Einfache Lösung:} Die Gleichungen nach jeder Rechnung umbenennen (Ia $\to$ Ib) und im Kopf behalten, dass man ab dann nichts mehr mit Ia rechnen darf.
\end{itemize}
\end{bla}

\clearpage

\begin{bsp}
  Ein vorgegebenes LGS soll gelöst werden. Eine Methode die auf jeden Fall zu einem Ergebnis führt ist die folgende:
  \begin{alignat*}{5}
  \intertext{Zuerst wird der x-Koeffizient in Ia zu einer Eins gemacht:}
    \LIN{Ia}{2}{-1}{4}{5}{Ib $=$ Ia$*\frac{1}{2}$}
    \LIN{IIa}{5}{2}{-10}{7}{}
    \LIN{IIIa}{12}{-9}{-8}{11}{}
  \intertext{Damit werden die anderen x-Terme aus der Gleichung subtrahiert:}
    \tag{Ib} x &\:\:& -\frac{1}{2}y &\:\:+& 2z &&&= \frac{5}{2} &&	\\[0pt]
    \LIN{IIa}{5}{2}{-10}{7}{IIb $=$ IIa$ - 5 * $Ib}
    \LIN{IIIa}{12}{-9}{-8}{11}{IIIb $=$ IIIa$ - 12 * $Ib}
  \intertext{Beachte, dass wir hier zwei Dinge gleichzeitig tun. Fehlergefahr!}
  \intertext{Im Folgenden werden die letzten beiden Gleichungen getauscht und ein y-Koeffizient zu einer Eins gemacht}
    \tag{Ib} x &\:\:& -\frac{1}{2}y &\:\:+& 2z &&&= \frac{5}{2} &&\\[0pt]
    \tag{IIb}  &\:\:& \frac{9}{2}y &\:\:& -20z &&&= -\frac{11}{2} &&\quad\text{| IIc $= - \frac{1}{3} * $IIIb}\\[0pt]
    \tag{IIIb}  &\:\:& -3y &\:\:& -32z &&&= -19 &&\quad\text{| IIIc = IIb}\\[0pt]
  \intertext{Mit dieser Gleichung kann man jetzt das andere y schön entfernen}
    \tag{Ib} x &\:\:& -\frac{1}{2}y &\:\:+& 2z &&&= \frac{5}{2} &&\\[0pt]
    \tag{IIc}  &\:\:& y &\:\:+& \frac{32}{3}z &&&= \frac{19}{3} &&\\[0pt]
    \tag{IIIc}  &\:\:& \frac{9}{2}y &\:\:& -20z &&&= -\frac{11}{2} && \quad\text{| IIId $=$ IIIc $ - \frac{9}{2} * $ IIc}\\[0pt]
  \intertext{Hier kann man $z$ einfach ablesen (teile IIId durch $-86$).}
    \tag{Ib} x &\:\:& -\frac{1}{2}y &\:\:+& 2z &&&= \frac{5}{2} &&\\[0pt]
    \tag{IIc}  &\:\:& y &\:\:+& \frac{32}{3}z &&&= \frac{19}{3} &&\\[0pt]
    \tag{IIId} &\:\:&   &\:\: & -68z &&&= -34 && \\[0pt]
  \end{alignat*}



  Also ist $z = \frac{1}{2}$, damit liefert Gleichung IIc:
  \[
  y + \frac{32}{3} * \frac{1}{2} = \frac{19}{3}
  \]
  Es folgt: $y=1$, damit liefert Gleichung Ib:
  \[
  x - \frac{1}{2} +\frac{2}{2} = \frac{5}{2}
  \]
  Damit ist $x=2$ und die Lösungsmenge:
  %
  %\tag{IIa} ax &\:\:+& by &\:\:+& cz &&&= d &&\quad\text{| e}\\[0pt]
  %\changeX{*\frac{5}{29}}

  \begin{eqnarray*}
    L = \left\{
    \begin{pmatrix}
      2 & 1 & \frac{1}{2}
    \end{pmatrix}
    \right\}
    \\
    \hline
  \changeX{x}
  \changeY{y}
  \changeZ{z}
\end{eqnarray*}
\end{bsp}

\reqnomode
