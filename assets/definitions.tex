%%
% Book metadata
\title{Mathe-Crashkurs 2018}
\author[Simon König, Joshua Fabian]{}
\publisher{Simon König, Joshua Fabian}

% -- GRAPHICX SETUP
%\setkeys{Gin}{width=\linewidth,totalheight=\textheight,keepaspectratio}
\graphicspath{{assets/img/}}

% -- FONT SETUP
\fvset{fontsize=\normalsize}

% Generates the index
\makeindex


% Define styles for bags and leafs
\tikzstyle{bag} = [text width=4em, text centered]
\tikzstyle{end} = [fill=white, circle, draw=black]

\usepackage[default]{sourcesanspro}
\usepackage[T1]{fontenc}

\makeatletter

\newcommand{\xeq}[2][]{\ensuremath{\overset{
  {\scriptscriptstyle{#2}}}{\underset{\mathclap{#1}}{=}}}}
\newcommand{\xleq}[2][]{\ensuremath{\overset{
  {\scriptscriptstyle{#2}}}{\underset{\mathclap{#1}}{\leq}}}}
%Tags nach links
\newcommand{\leqnomode}{\tagsleft@true\let\veqno\@@leqno}
\newcommand{\reqnomode}{\tagsleft@false\let\veqno\@@eqno}
\makeatother

\newcommand{\thistheoremname}{}
\newcommand{\R}{\mathbb{R}}
\newcommand{\N}{\mathbb{N}}
\newcommand{\Z}{\mathbb{Z}}

\newcommand{\ot}{\leftarrow}
\newcommand{\To}{\implies}

\newcommand{\e}{\text{\rmfamily e}}
\renewcommand{\i}{\text{\rmfamily i}}

%Nutzen in {alignat}{5}:{NR}{K1}{K2}{K3}{Ergebnis}
\newcommand{\LGL}[5]{#2 x\quad &+&  #3 y\quad &+&  #4 z\quad &&& = #5 \tag{#1}  && \quad}


%Malpunkte
\mathcode`\*="8000
{\catcode`\*\active\gdef*{\cdot}}
%Punkte zu Kommata
\DeclareMathSymbol{.}{\mathord}{letters}{"3B}

\swapnumbers
\theoremstyle{definition}
\newtheorem{df}{Definition}[chapter]
\newtheorem{sz}[df]{Satz}
\newtheorem{genericthm}[subsection]{\thistheoremname}
\newtheorem{bsp}[df]{Beispiel}
\newenvironment{bw}[1][\textbf{Beweis}]{\begin{proof}[#1]}{\end{proof}}

%Bedienung: \begin{bla}{Satz von Horst}
\newenvironment{bla}[1]
{\renewcommand{\thistheoremname}{#1}\begin{genericthm}\setlength\parindent{0em}~\par}{\end{genericthm}}


\tikzset{stage/.style = {draw,minimum width=15mm,minimum height=7mm},
      edgenode/.style = {font=\small,near start,fill=white}}

\setcounter{tocdepth}{1}
\setcounter{secnumdepth}{2}
\makeatletter
\@addtoreset{chapter}{part}
\makeatother

\titleformat{\chapter}%
  [display]% shape
  {\relax\ifthenelse{\NOT\boolean{@tufte@symmetric}}{\begin{fullwidth}}{}}% format applied to label+text
  {}% label
  {0pt}% horizontal separation between label and title body
  {\huge\itshape}% before the title body
  [\ifthenelse{\NOT\boolean{@tufte@symmetric}}{\end{fullwidth}}{}]% after the title body

\titleformat{\section}
  [hang]
  {\LARGE}
  {\thesection\ }
  {0pt}
  {}

\newenvironment{inhalt}
  {\begin{tcolorbox}[colback=red!5!white,colframe=red!75!black,title=In diesem Abschnitt:,arc=0mm]}{\end{tcolorbox}}

\newenvironment{koch}
  {\begin{tcolorbox}[colback=green!5!white,colframe=green!75!black,title=Kochrezept,arc=0mm]}{\end{tcolorbox}}
