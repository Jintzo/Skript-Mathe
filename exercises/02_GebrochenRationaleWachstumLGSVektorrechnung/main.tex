\documentclass[a4paper, oneside]{article}
\usepackage[utf8]{inputenc}
\usepackage[ngerman]{babel}
\usepackage[top=2.5cm, bottom=3cm, outer=2.5cm, inner=2.5cm, heightrounded]{geometry}
\usepackage{graphicx}
\usepackage{morefloats}
\usepackage{wrapfig}
\usepackage{hyperref}
\usepackage{cite}
\usepackage{siunitx}
\usepackage[default]{sourcesanspro}
\usepackage[T1]{fontenc}
\usepackage{url}
\usepackage{marginnote}
\usepackage[font=footnotesize]{caption}
\usepackage{color}
\usepackage{xcolor}
\usepackage{multicol}
\usepackage[fleqn]{mathtools}
\usepackage{amssymb}
\usepackage{wrapfig}
\usepackage[noindentafter]{titlesec}
\usepackage{fancyhdr}
\usepackage{lastpage}
\usepackage{comment}

%% LÖSUNGEN ANZEIGEN
\newif\ifshow
%\showtrue
\showfalse

%%%SECTIONING
\renewcommand*{\marginfont}{\noindent\rule{0pt}{0.7\baselineskip}\footnotesize}

\newcommand{\aufgabe}[1]{\subsection{#1}}
\newcommand{\loesung}[1]{\subsubsection{#1}}

\renewcommand{\theenumi}{\alph{enumi})}
\renewcommand{\labelenumi}{\text{\theenumi}}

\newcounter{aufgabe}
%\newenvironment{lsg}{\loesung}{}
\ifshow
  \newenvironment{lsg}{\loesung}{}
\else
  \excludecomment{lsg}
\fi

\newenvironment{inhalt}
  {\paragraph{Inhalt des Übungsblatts:}\itemize\let\origitem\item}
  {\enditemize\vspace{2em}}

\newcommand{\R}{\ensuremath\mathbb{R}}
\newcommand{\N}{\ensuremath\mathbb{N}}
\newcommand{\Z}{\ensuremath\mathbb{Z}}
\newcommand{\LM}{\ensuremath\mathbb{L}}
\newcommand{\intd}{\ensuremath\mathrm{d}}
\newcommand{\e}{\ensuremath\mathrm{e}}
\renewcommand{\d}{\,\mathrm{d}}
\newcommand{\stf}[1]{\ensuremath \left[ #1 \right]}

\newcommand{\cas}{\hfill (CAS)}

\everymath{\displaystyle}

%Malpunkte
\mathcode`\*="8000
{\catcode`\*\active\gdef*{\cdot}}

%SECTION
\titleformat{\section}
{\clearpage\setcounter{aufgabe}{0}\vspace{1em}\Large\raggedright\bfseries}
{}
{0pt}
{}

\titleformat{\subsection}[runin]
{\stepcounter{aufgabe}\vspace{1em}\normalfont\raggedright\bfseries}
{A\theaufgabe: }
{0pt}
{\ }

\titleformat{\subsubsection}[runin]
{\normalfont\raggedright\bfseries}
{Lösung \theaufgabe: }
{0pt}
{\ }


%FANCYHDR
\pagestyle{fancy}
\lhead{\small Simon König\\ Joshua Fabian}
\rhead{\small Mathecrashkurs 2018}
\cfoot{Seite \thepage\thinspace von\thinspace\pageref{LastPage}}
\lfoot{}
\renewcommand{\headrulewidth}{0.5pt}
\renewcommand{\footrulewidth}{0pt}

\title{Mathe-Crashkurs 2018 - Übungsblatt}
\date{\today}
\author{Simon König, Joshua Fabian}

\chead{\Large Übungsblatt 2}

\begin{document}
\begin{inhalt}
  \item Gebrochenrationale Funktionen
	\item Wachstum
	\item LGS-Rechnung, Vektorrechnung
\end{inhalt}

\aufgabe{Uneigentliches Flächenintegral}
\begin{enumerate}
	\item Berechnen Sie $\int\limits^\infty_0 e^{-x}\d x$
	\item	Bestimmen Sie $\int\limits^\infty_0 x^a\d x$ in Abhängigkeit von $a$ mit $a < -1$.
	\item Berechnen Sie $\lim\limits_{x\rightarrow -\infty} x^5\e^x$
\end{enumerate}

\aufgabe{Gebrochenrationale Funktionen}
\begin{enumerate}
	\item Gegeben sei die Funktion $f(x)=\frac{x}{x^2-9}$. Skizzieren Sie den Graphen und geben Sie den Definitionsbereich $D_f$ an.
	\item	Geben Sie die Funktionsgleichungen aller waagerechter oder senkrechter Asymptoten von $f(x)=\frac{4x^2+3}{x^2+5x}$
\end{enumerate}

\aufgabe{Wachstum}
\begin{enumerate}
	\item Bakterien vermehren sich durch Teilung, wobei sich eine Bakterienzelle durchschnittlich alle 10 Minuten teilt. Zum Zeitpunkt $t=0$ sei genau eine Bakterienzelle vorhanden.
	\begin{itemize}
		\item Wie viele Bakterien sind dann nach 1 Stunde, 2 Stunden, 6 Stunden, 12 Stunde bzw. 24 Stunden vorhanden?
		\item Finde eine Formel für die Anzahl $B(t)$ der Bakterien nach der Zeit $t$.
		\item Eine Bakterienzelle hat ein Volumen von ca. $2\times10^{-18}\mathrm{m}^3$ . Wie lange dauert es, bis die Bakterienkultur ein Volumen von $1\mathrm{m}^3$ bzw. $1\mathrm{km}^3$ einnimmt? Ist das Ergebnis plausibel?
	\end{itemize}
	\item	Angenommen, die Weltbevölkerung vermehrt sich nach der Formel $M(t)=M_0*\e^{\delta t}$. 1960 gab es ca. 3 Milliarden Menschen, 1995 etwa 5,6 Milliarden.
	\begin{itemize}
		\item Bestimme die Konstante $\delta$.
		\item Wieviel Prozent beträgt das jährliche Wachstum der Weltbevölkerung?
		\item Wann wird die Erde 15 Mrd. Einwohner haben, wenn die Bevölkerung im selben Tempo weiterwächst?
	\end{itemize}
	\item Eine Tasse kochendheißer Kaffee (100$^\circ$C) kühlt bei Zimmertemperatur (20$^\circ$C) in 10 Minuten auf 30$^\circ$C ab.
	\begin{itemize}
		\item Geben Sie eine Funktionsgleichung für die Temperatur des Kafffees an.
		\item Frau M mischt den Kaffee mit der gleichen Menge Milch aus dem Kühlschrank (4$^\circ$C). Sie hat zwei Möglichkeiten: die Milch sofort dazugeben, danach 3 Minuten warten oder die Milch erst nach 3 Minuten dazugeben.\\
Welche Temperatur hat der Milchkaffee in beiden Fällen?
\textit{(Hinweis: Die Temperatur der Mischung ist der Mittelwert der einzelnen Temperaturen: $T=\frac{(T_1+T_2)}{2}$.)}
	\end{itemize}
\end{enumerate}

\begin{lsg}{}
	\begin{enumerate}
		\item
		\item
		\item $0,208^\circ\mathrm{C} * \mathrm{min}^{-1}$\\
		Erst abkühlen, dann Milch dazu: $33,44^\circ\mathrm{C}$
		Variante 2: $37,14^\circ\mathrm{C}$
	\end{enumerate}
\end{lsg}

\aufgabe{LGS}
%Nutzen in {alignat}{5}:{NR}{K1}{K2}{K3}{Ergebnis}
\newcommand{\lgslinethree}[4]{#1 x\quad &+&  #2 y\quad &+&  #3 z\quad & = #4 \\}
\newcommand{\lgslinetwo}[3]{#1 x\quad &+&  #2 y\quad & = #3 \\}
Berechnen Sie die Lösungsmenge $\LM\subseteq\R$ der linearen Gleichungssysteme:
\begin{multicols}{2}
	\begin{enumerate}
		\item
		\begin{alignat*}{5}
			\lgslinethree{}{3}{}{1}
			\lgslinethree{3}{9}{4}{5}
			\lgslinethree{}{3}{2}{3}
		\end{alignat*}
		\item
		\begin{alignat*}{4}
			\lgslinethree{6}{}{}{31}
			\lgslinethree{3}{1}{4\frac 1 2}{19}
			\lgslinethree{-2}{-1}{0}{-15}
		\end{alignat*}
		\item
		\begin{alignat*}{4}
			\lgslinetwo{3}{2}{1}
			\lgslinetwo{1}{-2}{11}
			\lgslinetwo{-2}{}{5}
		\end{alignat*}
	\end{enumerate}
\end{multicols}



\aufgabe{Vektorrechnung}
\begin{enumerate}
	\item Berechne das Skalarprodukt der beiden Vektoren:
	\item Berechne $\alpha$ so, dass die beiden Vektoren orthogonal zueinander sind.
	\item
\end{enumerate}

\end{document}
